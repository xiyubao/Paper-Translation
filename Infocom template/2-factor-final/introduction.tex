\section{Introduction}
\label{sec:introduction}
Two-step authentication is a technique that performing an additional authentication process on the top of the traditional username-password authentication. In recent years, two-step authentication is gradually becoming a mandatory component for many information-sensitive websites, e.g., banking, healthcare, tax reporting and so on. It reported that over 74\% of organizations have adopted two-step authentication by 2018. The trend of gradual mandatory utilization of the two-step authentication is undeniable due to increasing incidents of username-password leakages resulted from rising numbers of attacks targeting public servers/databases~\cite{passwordleak}. 

However, designing a descent two-step authentication mechanism is far from being trivial than common beliefs. It is always difficult to seek a good balance between minimizing the human efforts to while guaranteeing the safety of the authentication mechanism. So far, the SMS-based authentication is widely used in the industrial world. After entering the username and the password, the user is required to enter a code which is sent to his cell phone as the text message. It seems that this mechanism can provide a promising security guarantee, yet Gelernter \emph{et al.} proposed a password reset MitM (PRMitM) attack luring a user to enter the SMS-based code that was sent from other trusted websites such as Gmail into the phishing website~\cite{gelernter2017password}. Their study found that over 63\% of the population is vulnerable to this attack. Also, according to a survey, 52\% of the population are not satisfied with the current SMS-based authentication since it is tedious and requires lots of human efforts. Researchers also proposed zero-effort sound-based authentication by comparing the ambient sound pieces collected from a user's smartphone and the web side to realize the two-step authentication~\cite{karapanos2015soundproof}. However, this authentication method was quickly broken by another research work which demonstrating two attacks that can specifically make the ambient sounds on the victim's side and the attacker's side identical~\cite{shrestha2016sounds}. Other well-known two-step authentication mechanisms such as face recognition, fingerprint matching, and sound recognition are all proven to be vulnerable to attacks exploiting LED light as adversarial examples, tangerine peel attack, and DolphinAttack respectively~\cite{zhou2018invisible}\cite{tangerinpeel}\cite{zhang2017dolphinattack}. We detail the limitations of the most well-known two-step authentication mechanisms in Section~\ref{sec:background}. From the above examples, we can see that constructing a safe two-step authentication mechanism which requires as little human effort as possible is difficult.

Having observed this urgent situation, in this paper, we propose a novel 2-step authentication mechanism, \name.\Name consists of three layers, the sensor layer, the ultrasound layer and the sound layer. In the sensor layer, sensor data are collected on both mobile side and web side and fed into the classifiers trained with our learning models to predict if the smartphone and the web are both roughly at the same location. This layer is designed mainly to defend against remote attackers. In the ultrasound layer, the server sends the smartphone a random code over TLS. Then the smartphone forwards the code to the web side through ultrasound. Lastly, the web side authenticates itself with the server using the code. This layer is designed mainly to defend against proximate attackers, i.e., the attackers locate at a short distance from the victim. In the sound layer, the server sends the web a random code. Then the web play a human audible sound, the smartphone recognize the code then sent it to the server. \Name barely requires any human efforts other than asking the user to put his cell phone close to the speaker on the web side. 

\textbf{Our Contributions.} The contributions made by our work are multi-fold
\begin{itemize}
  \item We analyze and summarize the limitations of the current well-known 
  two-step authentication mechanisms. As well as comparing them to \name.
  
  \item We propose a novel two-step authentication mechanism, \Name . We evaluated \name has a strong security, which achieved over 91.67\% of accuracy with different machine learning algorithms to defend against remote attacks and guaranteed to resist proximate attacks beyond 15 $cm$. In addition, \Name requires little human effort to complete.  
  
  \item We completely implemented the approach in section \uppercase\expandafter{\romannumeral3}, including the sensor layer, the ultrasound layer and the sound layer.
\end{itemize}

\textbf{Paper Organization.} 
The rest of the paper is organized as follows. Section~\ref{sec:background} presents the background knowledge. 
Section~\ref{sec:approach} details the design of \name. Section~\ref{sec:implementation} demonstrates the implementation of \name. Section~\ref{sec:Evaluation} presents the evaluations of \name. Section~\ref{sec:related} outlines the most related work. Section~\ref{sec:conclusion} concludes the paper.