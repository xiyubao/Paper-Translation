\section{Background}
\label{sec:background}
In this section, we mainly present the background knowledge of this paper. Firstly, we briefly introduce the basic concept of the two-step authentication. Then we present the threat model to the two-step authentication. Lastly, we summarize the limitations of the current well-known two-step authentication mechanisms in industrial field and academic field.

\subsection{Two-step Authentication}
Two-step authentication is a way to confirm a user's claimed identity by using what they know (password) and a second factor that they don't know. So far, the mostly adopted two-step authentication in the real world is the SMS-based authentication. This mechanism works as follows. Firstly, when setting up the 2-step verification, the user need to bind this phone by verifying the verification code or installing the app that can generate the digit number. Thereafter, each time the user log in to his account page, in addition to the password, the user also needs to enter the verification code sent by the server or generated by an application that is common to the user and the server to verify his identity. If the password and verification code are both correct, then the user can pass the verification.

\subsection{Threat Model}
We assume that an attacker, who intends to invade a victim's privacy protected by a two-step authentication, already compromises the victim's username and password. In this case, the attacker needs to crack the challenge of the authentication performed at the second stage. We consider two attack scenarios, remote attack and proximate attack. A remote attack refers to any attack that is taken place when an attacker locates far away from the victim. A proximate attack refers to any attack that is taken place when an attacker is close enough to the victim. 

\subsection{Analysis of the Current Two-step Authentication Mechanisms}
In this subsection, we summarize the limitations of some well-known two-step authentication mechanisms. SMS-based  authentication is probably the most widely adopted approach. Yet it is vulnerable to a remote MitM attack~\cite{gelernter2017password}. Face recognition as authentication is always widely utilized in industry~\cite{faceauth}. However, researchers found that careful setting up LED lights can fool the face recognition system~\cite{zhou2018invisible}. Fingerprint-based authentication is another widely used mechanism. Unfortunately, it is also found vulnerable. An attacker can spoof a victim's fingerprints using tangerine peel~\cite{tangerinpeel}. Sound-recognition authentication method is also adopted in many speaker-based systems such as Siri, Google Home and Amazon Echo~\cite{marley1988system}. It is proven vulnerable in some recent research~\cite{zhang2017dolphinattack}. SoundProof is zero-effort authentication mechanism which matches the ambient sounds collected from mobile and web to tell if they are at the same location~\cite{karapanos2015soundproof}. This method is proven vulnerable to a remote attack in which an attacker is able to create identical ambient sound from victim's mobile side and the attacker's web side~\cite{shrestha2016sounds}. Even though other authentications such as multi-touch authentication method has not been taken down, they require excessive human effort in order to complete. Table~\cite{tbl:limit} summarizes the limitations of these two-step authentication methods, as well as compares these methods with \name.

\begin{table}[]
\caption{Summary of limitations of the current two-step authentication mechanisms.}
\resizebox{\columnwidth}{!}{%
\label{tbl:limit}
\begin{tabular}{|c|c|c|c|}
\hline
                                                                           & \begin{tabular}[c]{@{}c@{}}Little\\ Human Effort\end{tabular} & \begin{tabular}[c]{@{}c@{}}Defend Against\\ Remote Attacks\end{tabular} & \begin{tabular}[c]{@{}c@{}}Defend Against\\ Proximate Attacks\end{tabular} \\ \hline
\begin{tabular}[c]{@{}c@{}}Face Recognition\\ Authentication\end{tabular}  & \cmark                                                             & \xmark                                                                       & \cmark                                                                          \\ \hline
\begin{tabular}[c]{@{}c@{}}Fingerprint\\ Authentication\end{tabular}       & \xmark                                                             & \xmark                                                                       & \cmark                                                                          \\ \hline
\begin{tabular}[c]{@{}c@{}}Sound Recognition\\ Authentication\end{tabular} & \xmark                                                             & \cmark                                                                       & \cmark                                                                          \\ \hline
SoundProof                                                                 & \cmark                                                             & \xmark                                                                       & \xmark                                                                          \\ \hline
\begin{tabular}[c]{@{}c@{}}Multi-Touch \\ Authentication\end{tabular}      & \xmark                                                             & \cmark                                                                       & \cmark                                                                          \\ \hline
\name                                                                      & \cmark                                                             & \cmark                                                                       & \cmark                                                                          \\ \hline
\end{tabular}
}
\end{table}

%\subsection{Adversary model}
%For the verification method of sending a short message with verification code from the server,xx(paper about the phishing website) present a feasible attack method as follow.
%
%\begin{enumerate}
%  
%  \item The premise of the 2-step verification attack is that the attacker has already known the user's account and password.
%  \item The attacker design a phishing website and sent it to the user, which appears to be a normal login interface with 2-step verification, but in fact sends the information entered by the visitor, including the verification code, to the attacker.
%  \item When a user logs into an phishing website, the attacker logs in to the user's Google account at the same time.
%  \item The server of Google will send a short message to the user's pre-bound mobile phone.
%  \item This paper points out that even if the researcher tells the tester that an attack will occur, most testers will not read the message containing the verification code carefully, and fill in the verification code directly on the phishing website.
%\end{enumerate}