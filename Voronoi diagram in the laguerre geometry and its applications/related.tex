\section{Related Work}
\label{sec:related}

\Name is primarily based on the acquisition and classification of sensor data and sonic communication. In this section, we present the relevant research in these two parts. 

\subsection{Sensor Information}

With the increase in mobile phone sensors, more and more research is beginning to extract user characteristics by sensor-based data for authentication. The research shows that sensor data can model user behaviour\cite{kayacik2014data} for authentication. which can be a accelerometer sensor\cite{hoang2015gait}, several motion sensors\cite{shen2018performance} or multi-sensors\cite{lee2015multi}. Most of them are used to authenticate whether the person using the phone is the user himself\cite{li2018sensor} based on user's behavioral characteristics\cite{lee2016implicit}.

The sensor data can also be used in network authentication, Shen \emph{et al.} present a multi-party security protocol that incorporates biometric-based authentication\cite{shen2018usable}, because the current security mechanisms are inconvenient for most mobile devices that typically have a much smaller form factor than PCs.




\subsection{Sonic communication}
Recently, sonic communication is very popular, especially in many embedded platform (Android, iOS, linux, etc.). Sound wave technology theoretically determines that it is suitable for short distance and small amount of information transmission. SinVoice is the first commercial sonic codec library on the market that provides sonic codec\cite{sinvoice}. NearBytes is a contactless communication technology that ensures successful communication between devices\cite{nearbytes}.

Some well-known applications such Alipay now support sonic payment\cite{alipay}. Ita{{\'u}} bank ran a commercial where movie tickets were given from radio directly to mobile phones\cite{tickets}. Many embedded devices are now using sonic communication technologies, such smart routers, cameras, car electronics (such driving recorders), etc.


